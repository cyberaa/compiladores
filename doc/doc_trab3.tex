\documentclass[a4paper,10pt]{article}

\usepackage[a4paper,left=1.0cm,right=1.0cm,top=2.0cm]{geometry}

\usepackage[utf8]{inputenc}
\usepackage{listings}
\usepackage{graphicx}
\usepackage[pdftex]{hyperref}
\usepackage{color}
\usepackage{textcomp}
\usepackage{float}

\definecolor{listinggray}{gray}{0.9}
\definecolor{lbcolor}{rgb}{0.9,0.9,0.9}

\lstset{
	backgroundcolor=\color{lbcolor},
	tabsize=2,
	rulecolor=,
	language=C,
        basicstyle=\scriptsize,
        upquote=true,
        aboveskip={1.5\baselineskip},
        columns=fixed,
        showstringspaces=false,
        extendedchars=true,
        breaklines=true,
        prebreak = \raisebox{0ex}[0ex][0ex]{\ensuremath{\hookleftarrow}},
        frame=single,
        showtabs=false,
        showspaces=false,
        showstringspaces=false,
        identifierstyle=\ttfamily,
        keywordstyle=\color[rgb]{0,0,1},
        commentstyle=\color[rgb]{0.133,0.545,0.133},
        stringstyle=\color[rgb]{0.627,0.126,0.941},
}

%opening
\title{3º Trabalho de Compiladores - Analisador Sintático}
\author{André Siviero e Juan França}


\begin{document}

\maketitle

\begin{abstract}
Documentação referente à terceira parte do trabalho de compiladores, que consiste em um analisador semântico.
\end{abstract}

\section{Introdução}
O objetivo deste trabalho é estabelecer um analisador semântico para a linguagem PC (Poor C). O analisador semântico é a terceira parte de um compilador,
e sua função é analisar a coerência dos elementos contidos em um arquivo de entrada (código fonte). 

Utilizando a ferramenta Bison para gerar um analisador sintático, a tarefa foi facilitada. Nesta documentação, são descritas as regras da linguagem e as dificuldades encontradas pela dupla.

\section{Objetivos}
Basicamente, este terceiro trabalho consistia em verificar:
\begin{enumerate}
\item Se as variáveis utilizadas foram declaradas;
\item Se há variáveis redeclaradas;
\item Se há variáveis declaradas e não utilizadas;
\item Se os tipos associados às variáveis e ao valor associado são compatíveis;
\item Se o número de argumentos (aridade) de uma função está correto;
\item Se o tipo associado ao valor de retorno de uma função está correto;
\item Se os tipos associados aos argumentos de uma função estão corretos.
\end{enumerate}

Para solucionar isto foi implementado dois TADs: tabela Hash e lista genérica. O tamanho máximo utilizado para tabela foi de 997. A função que gera as keys está descrito a seguir:

\begin{lstlisting}
int sum_ascii(char *string) {
  int i = 0;
  int ascii = 0;

  for(i=0;string[i]!='\0';i++) {
    ascii += string[i];
  }
  return ascii;
}
\end{lstlisting}

Antes de explicar o que foi utilizado para solucionar cada item, será feita uma breve introdução de como funciona o código.

\section{Como funciona?}

\subsection{Estruturas}
Temos dois tipos básicos de estruturas para ajudar a concluir os objetivos acima. Estes são descritos abaixo:
\subsubsection{s\_funcao}

Cada função é definida da seguinte forma:

\begin{lstlisting}
struct  {
  char nome[30];
  int aridade;
  int tipo_retorno;
  list parametros;
} typedef s_funcao;
\end{lstlisting}

As funções pré-definidas (min,max,printf e scanf) são inseridas na hash no início do programa. E a lista de parâmetros destas é checada no momento de execução pois os tipos são
variáveis.

\subsubsection{s\_variavel}
Cada variável é definida da seguinte forma:
\begin{lstlisting}
 struct  {
  char nome[30];
  void *valor;
  int tipo;
  char escopo[30];
  int lineDeclared;
  int used;
} typedef s_variavel;
\end{lstlisting}

O atribuito 'used' serve para determinar se uma variável foi usada ou não e o 'lineDeclared' armazena a linha da declaração da variável. Será explicado com mais detalhes no decorrer do relatório.

\subsection{Declaração de funções}
Na Poor C, só pode existir declaração de funções antes da main. Toda vez que tem-se uma declaração, esta é inserida na tabela hash de funções, não há sobrecarga de funções. Portanto, se ao tentar
inserir uma função, já existir uma função com o mesmo nome na hash ocorre erro. Os parâmetros da funções são inseridos na tabela de variáveis com o tipo definido na declaração.
\subsection{Chamada de função}
Quando ocorre uma chamada de função, busca-se na hash o nome da função. Caso não exista a função, é retornado erro.
\subsection{Declaração de variáveis}
Similarmente a declaração de funções, a declaração de variáveis executa o mesmo comando: tenta inserir na tabela hash de variáveis. Caso já tenha uma variável com o mesmo escopo ocorreu uma
redeclaração de variáveis.
\subsection{Atribuição de variáveis}
Para atribuir valores as variáveis, é necessário procurar na tabela hash de variáveis o nome da variável assoaciado ao escopo. A seguir deve-se verificar se o tipo da expressão
é compatível com o tipo da variável. Para isso, temos a função ilegível returnTypeExprList que atua sobre uma lista de tipos e operadores retornando o tipo resultante.
\subsection{Atrib/Ident}
A variável 'ident' guarda o valor do último identificador. A variável global 'atrib' guarda é um comando simples do código fonte, na maioria das vezes, equivalente a uma linha de código.
É necessário zerar 'atrib' a cada comando terminado.
\subsection{ExprList}

\section{Se as variáveis utilizadas foram declaradas}
Checa-se na tabela hash de variáveis se o nome da variável e com o respectivo escopo existe ou o nome da variável com escopo 'global' existe. 
Caso exista é porque já foi declarada. Isto também checa que parâmetros da função usados dentro dela também são variáveis.

\section{Se há variáveis redeclaradas}
Caso similar ao anterior, caso em uma declaração de variável esta já existe(com mesmo nome e mesmo escopo) deve retornar erro.
\section{Se há variáveis declaradas e não utilizadas}
Para solucionar este tópico foi necessário adicionar os atributos a variável 'lineDeclared' e 'used'. Se no final do programa, existir alguma varíavel na tabela hash de variáveis
com used com valor igual a zero significa que a variável não foi utilizada. O atributo 'LineDeclared' guarda a linha que foi declarada para retornar junto ao erro.
\section{Se os tipos associados às variáveis e ao valor associado são compatíveis}

\section{Se o número de argumentos (aridade) de uma função está correto}
Em cada declaração de função a aridade é conhecida. Basta checar se na tabela hash de funções o valor da aridade é o mesmo da função que está sendo checada.
\section{Se o tipo associado ao valor de retorno de uma função está correto}

\section{Se os tipos associados aos argumentos de uma função estão corretos}

\section{Exemplos}
Nos exemplos, exploramos diversas situações para os diversos tipos de usuários. Foram criadas 10 situações, 5 situações corretas e 5 
situações erradas. Um teste com situações bizarras e mais difíceis também foi incluído.
\end{document}
