\documentclass[a4paper,10pt]{article}
\usepackage[utf8x]{inputenc}

%opening
\title{1º Trabalho de Compiladores - Analisador Léxico}
\author{André Siviero e Juan França}

\begin{document}

\maketitle

\begin{abstract}
Documentação referente à primeira parte do trabalho de compiladores, que consiste em um analisador léxico
\end{abstract}

\section{Introdução}
O objetivo deste trabalho é estabelecer um analisador léxico para a linguagem C. O analisador léxico é a primeira parte de um compilador,
e sua função é classificar elementos contidos num arquivo de entrada (fonte). Assim, nesta primeira etapa, busca-se encontrar palavras
reservadas de linguagem, operadores, identificadores, número e símbolos especiais da linguagem, como delimitadores de bloco e comentários.

Utilizando a ferramenta Flex e expressões regulares, a tarefa foi facilitada. Nesta documentação, são descritas as expressões regulares que
deram origem ao arquivo .l.
\section{Classificação das entradas}
  \subsection{Palavras Reservadas}
  A linguagem C define as seguintes palavras reservadas:
  \begin{verbatim}
   auto, break, case, char, const, continue, default, do, double, 
   else, enum, extern, float, for, goto, if, int, long, register, 
   return, short, signed, sizeof, static, struct, switch, typedef, 
   union, unsigned, void, volatile, while
  \end{verbatim} 
  
  O trecho de código correspondente a elas é:
  \begin{verbatim}
auto {printf("%s -> PALAVRARESERVADA\n",yytext);}
break {printf("%s -> PALAVRARESERVADA\n",yytext);}
case {printf("%s -> PALAVRARESERVADA\n",yytext);}
char {printf("%s -> PALAVRARESERVADA\n",yytext);}
const {printf("%s -> PALAVRARESERVADA\n",yytext);}
continue {printf("%s -> PALAVRARESERVADA\n",yytext);}
default {printf("%s -> PALAVRARESERVADA\n",yytext);}
do {printf("%s -> PALAVRARESERVADA\n",yytext);}
double {printf("%s -> PALAVRARESERVADA\n",yytext);}
else {printf("%s -> PALAVRARESERVADA\n",yytext);}
enum {printf("%s -> PALAVRARESERVADA\n",yytext);}
extern {printf("%s -> PALAVRARESERVADA\n",yytext);}
float {printf("%s -> PALAVRARESERVADA\n",yytext);}
for {printf("%s -> PALAVRARESERVADA\n",yytext);}
goto {printf("%s -> PALAVRARESERVADA\n",yytext);}
if {printf("%s -> PALAVRARESERVADA\n",yytext);}
int {printf("%s -> PALAVRARESERVADA\n",yytext);}
long {printf("%s -> PALAVRARESERVADA\n",yytext);}
register {printf("%s -> PALAVRARESERVADA\n",yytext);}
return {printf("%s -> PALAVRARESERVADA\n",yytext);}
short {printf("%s -> PALAVRARESERVADA\n",yytext);}
signed {printf("%s -> PALAVRARESERVADA\n",yytext);}
sizeof {printf("%s -> PALAVRARESERVADA\n",yytext);}
static {printf("%s -> PALAVRARESERVADA\n",yytext);}
struct {printf("%s -> PALAVRARESERVADA\n",yytext);}
switch {printf("%s -> PALAVRARESERVADA\n",yytext);}
typedef {printf("%s -> PALAVRARESERVADA\n",yytext);}
union {printf("%s -> PALAVRARESERVADA\n",yytext);}
unsigned {printf("%s -> PALAVRARESERVADA\n",yytext);}
void {printf("%s -> PALAVRARESERVADA\n",yytext);}
volatile {printf("%s -> PALAVRARESERVADA\n",yytext);}
while {printf("%s -> PALAVRARESERVADA\n",yytext);}

  \end{verbatim} 

  \subsection{Identificadores}
  Em C, identificadores são definidos como uma letra (maiúscula ou minúscula) ou um underscore (\_) seguido de letras, underscores ou
  números. A expressão regular que define um identificador em C é dada por:
  \begin{verbatim}
[_a-zA-Z][_a-zA-Z0-9]*	{ printf("%s -> IDENTIFICADOR\n",yytext);}
  \end{verbatim}
  

\subsection{Inteiros e Floats}
  Números inteiros são definidos como uma sequência de dígitos seguidos, opcionalmente, de um expoente positivo. Para reconhecê-los,
  definimos:
  \begin{verbatim}
digitos [0-9]+ 
expoente_positivo [eE]{digitos}
inteiro {digitos}+{expoente_positivo}?
\end{verbatim}

Floats são definidos como uma sequência de dígitos, seguidos de um ponto, outra sequência de dígitos e um expoente. Floats foram
reconhecidos como:
\begin{verbatim}
expoente [eE]-?{digitos} 
fracao "."{digitos}
float {digitos}{fracao}?{expoente}?
\end{verbatim}  


  \subsection{Operadores aritméticos}
  Os operadores aritméticos são (+  -  *  /  ++  -- \% $\ll$ $\gg$). O trecho de código correspondente a eles é:
  \begin{verbatim}
"+" {printf("%s -> OPERADORARITMETICO\n",yytext);}
"-" {printf("%s -> OPERADORARITMETICO\n",yytext);}
"*" {printf("%s -> OPERADORARITMETICO\n",yytext);}
"/" {printf("%s -> OPERADORARITMETICO\n",yytext);}
"%" {printf("%s -> OPERADORARITMETICO\n",yytext);}
"++" {printf("%s -> OPERADORARITMETICO\n",yytext);}
"--" {printf("%s -> OPERADORARITMETICO\n",yytext);}
">>" {printf("%s -> OPERADORARITMETICO\n",yytext);}
"<<" {printf("%s -> OPERADORARITMETICO\n",yytext);}
  
  \end{verbatim}


  \subsection{Operadores Relacionais}
  Os operadores relacionais são ($>$, $>=$, $<$, $<=$, $==$, $!=$). O trecho de código correspondente a eles é:
  \begin{verbatim}
">" {printf("%s -> OPERADORRELACIONAL\n",yytext);}
"<" {printf("%s -> OPERADORRELACIONAL\n",yytext);}
">=" {printf("%s -> OPERADORRELACIONAL\n",yytext);}
"<=" {printf("%s -> OPERADORRELACIONAL\n",yytext);}
"==" {printf("%s -> OPERADORRELACIONAL\n",yytext);}
"!=" {printf("%s -> OPERADORRELACIONAL\n",yytext);}
  \end{verbatim}

  \subsection{Operadores Lógicos}
  Os operadores lógicos são (\&\& $||$ ! \^). O trecho de código correspondente a eles é:
  \begin{verbatim}
"&&" {printf("%s -> OPERADORLOGICO\n",yytext);}
"||" {printf("%s -> OPERADORLOGICO\n",yytext);}
"&" {printf("%s -> OPERADORLOGICO\n",yytext);}
"|" {printf("%s -> OPERADORLOGICO\n",yytext);}
"^" {printf("%s -> OPERADORLOGICO\n",yytext);}
"!" {printf("%s -> OPERADORLOGICO\n",yytext);}
  \end{verbatim}

  \subsection{Símbolos Especiais e Operadores de Atribuição}
  Os símbolos especiais são ( ( ) , ; : \{ \} \# ' $\backslash$ \textquotedblleft . ). O trecho de código correspondente a eles e aos operadores de atribuição é:
  \begin{verbatim}
"(" {printf("%s -> SIMBOLOESPECIAL\n",yytext);}
")" {printf("%s -> SIMBOLOESPECIAL\n",yytext);}
"," {printf("%s -> SIMBOLOESPECIAL\n",yytext);}
";" {printf("%s -> SIMBOLOESPECIAL\n",yytext);}
":" {printf("%s -> SIMBOLOESPECIAL\n",yytext);}
"{" {printf("%s -> SIMBOLOESPECIAL\n",yytext);}
"}" {printf("%s -> SIMBOLOESPECIAL\n",yytext);}
"#" {printf("%s -> SIMBOLOESPECIAL\n",yytext);}
"'" {printf("%s -> SIMBOLOESPECIAL\n",yytext);}
"\"" {printf("%s -> SIMBOLOESPECIAL\n",yytext);}
"\\" {printf("%s -> SIMBOLOESPECIAL\n",yytext);}
"." {printf("%s -> SIMBOLOESPECIAL\n",yytext);}
"=" {printf("%s -> ATRIBUICAO\n",yytext);}
"+=" {printf("%s -> ATRIBUICAO\n",yytext);}
"-=" {printf("%s -> ATRIBUICAO\n",yytext);}
"*=" {printf("%s -> ATRIBUICAO\n",yytext);}
"/=" {printf("%s -> ATRIBUICAO\n",yytext);}
"&=" {printf("%s -> ATRIBUICAO\n",yytext);}
"^=" {printf("%s -> ATRIBUICAO\n",yytext);}
"|=" {printf("%s -> ATRIBUICAO\n",yytext);}
">>=" {printf("%s -> ATRIBUICAO\n",yytext);}
"<<=" {printf("%s -> ATRIBUICAO\n",yytext);}

  \end{verbatim}

\subsection{Comentários}
  Comentários devem ser ignorados pelo analisador léxico. Comentário de uma única linha são simples, porém os comentários em bloco
  são um pouco mais complexos. O código utilizado para reconhecê-los foi:
  \begin{verbatim}
comentario_linha "//"[^\n]*
comentario_bloco "/""*"+([^*/][^*]*"*"+)*"/"
\end{verbatim}
Além disso, quando encontra-se um bloco de comentário, é feita uma pequena correção na variável lines, que conta o número de linhas
do arquivo, para incluir as quebras de linha que estejam dentro do bloco de comentário:
\begin{verbatim}
{comentario_bloco} {
  strcpy(comment,yytext);	
  for(i=0; i<strlen(comment); i++) {
    if(comment[i] == '\n') lines++;
  }
}
\end{verbatim}

\section{Exemplos}
Nos exemplos, buscamos identificar os diversos operadores, identificadores e demais características do input. Também prestamos atenção
aos blocos de comentários que foram problemáticos. 
\end{document}
